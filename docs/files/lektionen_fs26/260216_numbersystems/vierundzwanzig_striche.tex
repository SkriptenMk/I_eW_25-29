\documentclass[tikz]{standalone}

% Schriftart-Vorgabe: Helvetica-Ersatz
\usepackage{tgheros}
\renewcommand{\familydefault}{\sfdefault}

% Notwendige TikZ-Bibliotheken
\usetikzlibrary{positioning, arrows.meta, calc, shapes, decorations.pathmorphing}

\begin{document}

\tikzset{
    % Stil für den handschriftlichen Look
    handdrawn/.style={
        draw=blue!70!black,
        line width=1.2pt,
        line cap=round,
        % Erzeugt die leichte Unregelmäßigkeit in der Linie
        decoration={random steps, segment length=2mm, amplitude=0.3pt},
        decorate
    }
}

\begin{tikzpicture}

    % Schleife für die ersten 4 Fünfergruppen (4 * 5 = 20 Striche)
    \foreach \g in {0, 1, 2, 3} {
        \begin{scope}[xshift=\g * 1.5cm] % Horizontaler Abstand zwischen den Gruppen
            
            % 4 vertikale Striche pro Gruppe
            \foreach \x in {0, 0.2, 0.4, 0.6} {
                \draw[handdrawn] 
                    (\x + rand*0.02, 0 + rand*0.05) -- (\x + rand*0.02, 1 + rand*0.05);
            }
            
            % Der diagonale Querstrich für den 5. Strich
            \draw[handdrawn] (-0.1 + rand*0.05, 0.8 + rand*0.05) -- (0.7 + rand*0.05, 0.2 + rand*0.05);
            
        \end{scope}
    }

    % Die letzte Gruppe (nur 4 Striche ohne Querstrich = insgesamt 24)
    \begin{scope}[xshift=6cm] % Positionierung nach den 4 Blöcken
        \foreach \x in {0, 0.2, 0.4, 0.6} {
            \draw[handdrawn] 
                (\x + rand*0.02, 0 + rand*0.05) -- (\x + rand*0.02, 1 + rand*0.05);
        }
    \end{scope}

\end{tikzpicture}

\end{document}